Inicialmente, o projeto tem como objetivo buscar uma solução para as deficiências minerais encontradas na mexeriqueira (Citrus reticulata), especificamente na variedade conhecida como mexerica, apresentando uma alternativa ao problema por meio do uso de tecnologia. Assim, elaboramos um projeto cuja ideia central é criar uma aplicação capaz de identificar, em tempo real, as deficiências minerais pelas quais a planta está passando. Isso permitirá produzir um diagnóstico preciso, facilitando o processo de recuperação da planta, evitando a perda de frutos e otimizando o tempo dos produtores, suprindo a necessidade de uma identificação rápida e eficiente diretamente no campo.

Durante a pesquisa, identificamos algumas limitações relacionadas às condições do solo e à abrangência geográfica do estudo, que é restrito a alguns países. A deficiência de minerais nas plantas está diretamente associada à acidez do solo. Nosso estudo tem como objetivo identificar, especificamente, deficiências de manganês e cobre em plantas de Citrus reticulata (mexerica) por meio de análise foliar. Pesquisas indicam que uma planta com deficiência de manganês geralmente não apresenta deficiência de cobre, e vice-versa. Análises mais aprofundadas mostram que essas deficiências estão intimamente ligadas ao pH do solo: a carência de manganês tende a ocorrer em solos mais ácidos, enquanto a deficiência de cobre é mais comum em solos alcalinos \cite{ConclusãoMicroN, ConclusãoCobre}.

Futuramente, o projeto busca desenvolver um aplicativo para dispositivos móveis, como celulares e tablets, que empregará visão computacional, aprendizado profundo (\textit{deep learning}) e redes neurais convolucionais para identificar, de forma rápida e eficiente, a presença de uma das duas deficiências minerais abordadas neste estudo.