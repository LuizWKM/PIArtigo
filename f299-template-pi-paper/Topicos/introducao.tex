É vivido um mundo onde não há comida o suficiente para todos e outros problemas como guerras e pobreza o assolam, esse é um mal que há a muito tempo, e foi em 1945 que a Organização das Nações Unidas (ONU) foi criada para tentar combater os problemas que afligem o planeta. Como uma forma de auxiliar essa resolução é em setembro de 2015, que a ONU cria os Objetivos de Desenvolvimento Sustentável (ODS), que foram definidos na Agenda 2030. É um movimento global que tem como meta acabar com a pobreza, proteger o meio ambiente e o clima, erradicar a fome e trazer a paz e a prosperidade as pessoas de todos os lugares. Neste projeto é trabalhado com o objetivo de diminuir as possíveis perdas de citrus reticulata(mexerica), que são providas , e é por conta disso que nosso projeto contempla a ODS 2, Fome zero e agricultura sustentável, com isso em mente nosso foco é garantir que as mexericas possam ter uma produtividade maior \cite{IntroduçãoGreening}.

Atualmente, a taxa de produção de citros está diminuindo em todo o mundo, isso acontece por condições adversas de tempo ou por meio de pragas ou doenças que os citros podem vir a contrair, algumas das doenças são em nível global, como por exemplo a Phytophthora citrophthora, Xylella fastidiosa, cancro cítrico e o greening também conhecido como HLB e huanglongbing. No continente da América do Sul e América do Norte temos uma grande incidência do greening, isto pode ser visto por exemplo nos Estados Unidos que ano passado teve uma safra menor do que a normal por conta do greening e furacões \cite{IntroduçãoEUAProblemas}.

O Brasil também tem sua taxa de infecção por greening, esse greening é uma doença que ataca todos os tipos de citros e quando infectada a planta não tem cura, quando a planta infectada é nova elas não chegam a produzir frutos e as plantas adultas sofrem grande queda prematura dos frutos e definham ao longo do tempo. Uma das bactérias que causa essa doença no Brasil é a Candidatus Liberibacter asiaticus, Representando 99\% dos casos de greening , ela é a principal responsável pela doença. Essa bactéria é transmitida pelo psilídeo Diaphorina citri, também conhecido como Psilídeo-asiático-dos-citros, é um inseto de coloração branca acinzentada com manchas escuras nas asas, tem um comprimento de 2 a 3 mm e é muito frequente em pomares nas épocas de brotação das plantas. O estado de São Paulo também é habitado pelo greening, isso pode ser visto pelo Programa Nacional de Prevenção e Controle ao HLB (PNCHLB), que obriga a eliminação de plantas menores de oito anos que tenham contraído o HLB, ou seja, este problema também assola o vale do ribeira do estado de São Paulo \cite{IntroduçãoGreening}.

O HLB, popularmente conhecida como greening, tem como local de origem a Ásia há mais de 100 anos. Foi identificado no Brasil em 2004, nas regiões Centro e Leste do Estado de São Paulo, espalhando para todas as regiões citrícolas paulistas e pomares de Minas Gerais e Paraná, além de espalhar para a Argentina e Paraguai \cite{IntroduçãoGreening}.  

Por conta do greening ser uma doença que não tem cura, e após o descobrimento dela em um campo ser necessário a sua retirada para diminuir o risco de contaminações para as plantas ao redor, não há muito que possa ser feito para salvar a planta. Os sintomas do greening são folhas amareladas quando jovens e mosqueadas, apresentam pintas ou malhas escuras, quando maduras, além de que as folhas afetadas tendem a cair e em seu lugar nascer uma folha em posição vertical. Por conta disso a distinção entre o greening e outras doenças ou condições que tenham sintomas semelhantes é importante, para não se matar uma planta que pode ser salva. Duas das condições por deficiência de nutrientes que tem uma semelhança com os sintomas do greening são a deficiência de cobre e a de manganês. A deficiência de cobre tem como sintomas as folhas do fruto do novo ciclo de crescimento pequenas, deformadas e com nervuras verdes bem definidas sobre um fruto claro e as plantas podem apresentar folhas gigantes, bolsas de goma nos ramos novos e na casca dos frutos. Já a deficiência de manganês manifestam seus sintomas nas folhas novas, com um tamanho praticamente normal, perda de brilho e clorose, uma cor amarela esverdeada, entre as nervuras que permanecem verdes \cite{IntroduçãoGreening, IntroduçãoDeficiencias}.

São essas duas condições, as deficiências de cobre e manganês que nosso projeto busca analisar nas folhas da citrus reticulata. Foram escolhidas essas duas condições por terem um grau de semelhança com alguns dos sintomas do greening, como o amarelamento da folha e o crescimento diferente da folha. Nosso projeto busca verificar por meio de uma fota se a folha apresenta os sintomas de uma dessas duas deficiências de nutriente ou não. E com base nisso o agricultor pode aplicar as devidas medidas para combater os problemas, por meio da verificação do estado nutricional do solo em que a planta está plantada e com isso colocar os fertilizantes adequados no solo de maneira certa e balanceada.

Atualmente a Inteligência Artificial (IA) vem sendo utilizada em diversas área do mercado de trabalho, desde do ramo de saúde até os anúncios que vemos diariamente, sendo utilizada para aumentar a eficiência e qualidade, dando um resultado com maior exatidão em um número grande de situações. O objetivo das IA é pensar de forma autônoma e com respostas precisas para cada situação, sem ser necessário uma intervenção humana. Fazendo com que sua utilização em grandes empresas, hospitais e outros usuários que precisam automatizar algo, seja a cada dia maior e mais pronunciado.

O nosso sistema vai fazer uso  da IA para identificar a deficiência de manganês e cobre na folha da mexerica, a IA escanearia a foto e procuraria os sintomas das duas deficiências na folha, poderia ser uma folha com clorose, ou uma folha com coloração verde forte. Com base nisso ela identificaria e poderia retornar se a folha tem umas das duas deficiências, ou nenhuma delas.

A parte da IA que nosso projeto utilizara será a visão computacional, que tem como intuito permitir que a máquina veja visualmente e com base nisso identifique diversos dados pela imagem. Para poder ser utilizado a visão computacional primeiramente é necessário duas tecnologias chaves, rede neural convolucional e aprendizado de máquina profundo, popularmente conhecido como Deep Learning. São com essas duas ferramentas que se torna possível a máquina identificar certos padrões. A rede neural convolucional é usada em casos que há a necessidades de visão computacional, ela é inspirada na organização hierárquica do córtex visual humano,  construída com camadas intrinsecamente conectadas de estruturas de neurônios \cite{IntroduçãoRedeNeural}. No caso deep learning é uma tecnologia que chegou depois de machine learning, e é uma rede neural de multicamadas. Com deep learning é necessário apenas dizer se na imagem tem uma folha de citrus reticulata saudável ou uma com deficiência, e após enviar muitas imagens falando o que são cada uma ele iria descobrir o padrão. Já machine learning é necessário você por meio de programação descrever como é folha da planta saudável, falando sobre como seriam os pixels em uma imagem do padrão que você quer que a maquina reconheça \cite{IntroduçãoDeepLearning}.   