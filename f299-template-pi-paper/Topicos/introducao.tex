Vivemos em um mundo onde a comida não é suficiente para todos, e problemas como guerras e pobreza continuam a assolar diversas regiões. Esse mal perdura há muito tempo, e, em 1945, a Organização das Nações Unidas (ONU) foi criada para tentar combater esses desafios globais. Como parte dos esforços para enfrentar essas questões, a ONU lançou, em setembro de 2015, os Objetivos de Desenvolvimento Sustentável (ODS), definidos na Agenda 2030. Esse movimento global tem como metas principais erradicar a pobreza, proteger o meio ambiente e o clima, acabar com a fome e promover paz e prosperidade em todo o mundo. No presente projeto, buscamos reduzir as possíveis perdas de Citrus reticulata (mexerica), o que se alinha ao ODS 2 — Fome Zero e Agricultura Sustentável. Nosso foco é garantir uma maior produtividade das mexericas, contribuindo assim para a segurança alimentar e a sustentabilidade agrícola.\cite{IntroduçãoGreening}.

Atualmente, a produção de citros está em declínio em todo o mundo, devido a condições climáticas adversas, além de pragas e doenças que afetam essas culturas. Algumas dessas doenças têm alcance global, como Phytophthora citrophthora, Xylella fastidiosa, o cancro cítrico e o greening, também conhecido como HLB (Huanglongbing). Nas Américas do Sul e do Norte, há uma alta incidência do greening. Nos Estados Unidos, por exemplo, a colheita do último ano foi menor do que o esperado, resultado da combinação de infestações por greening e furacões \cite{IntroduçãoEUAProblemas}.

No Brasil, a taxa de infecção por greening também é significativa. Essa doença ataca todas as variedades de citros e, uma vez infectada, a planta não tem cura. Plantas jovens contaminadas geralmente não chegam a produzir frutos, enquanto plantas adultas sofrem queda prematura de frutos e, eventualmente, definhamento. Uma das principais bactérias causadoras do greening no Brasil é a Candidatus Liberibacter asiaticus, responsável por 99\% dos casos registrados no país. A bactéria é transmitida pelo psilídeo Diaphorina citri, também conhecido como psilídeo-asiático-dos-citros, um inseto de coloração branco-acinzentada com manchas escuras nas asas, que mede entre 2 e 3 mm. Esse inseto é especialmente frequente nos pomares durante as épocas de brotação das plantas. O estado de São Paulo é uma das regiões mais afetadas pelo greening, conforme evidenciado pelo Programa Nacional de Prevenção e Controle ao HLB (PNCHLB), que exige a eliminação de plantas com menos de oito anos contaminadas pelo HLB, o que inclui áreas como o Vale do Ribeira, também no estado de São Paulo \cite{IntroduçãoGreening}.

O HLB, popularmente conhecido como greening, teve origem na Ásia há mais de 100 anos. No Brasil, a doença foi identificada em 2004, nas regiões Centro e Leste do Estado de São Paulo, e desde então se espalhou para todas as áreas citrícolas do estado, bem como para pomares em Minas Gerais e Paraná. Além disso, a doença se expandiu para outros países da América do Sul, como Argentina e Paraguai \cite{IntroduçãoGreening}. 

Como o greening é uma doença incurável, a remoção imediata das plantas infectadas é necessária após a detecção em um campo, a fim de reduzir o risco de contaminação para as plantas ao redor. Infelizmente, pouco pode ser feito para salvar plantas infectadas. Os sintomas do greening incluem folhas amareladas quando jovens, mosqueadas com pintas ou malhas escuras quando maduras. Além disso, as folhas afetadas tendem a cair, sendo substituídas por outras em posição vertical. Por isso, a distinção entre o greening e outras doenças ou condições com sintomas semelhantes é crucial, para evitar a eliminação de plantas que poderiam ser salvas. Duas das condições causadas por deficiência de nutrientes que apresentam semelhança com os sintomas do greening são a deficiência de cobre e a de manganês. A deficiência de cobre manifesta-se por folhas pequenas e deformadas no novo ciclo de crescimento, com nervuras verdes bem definidas sobre uma coloração mais clara. As plantas também podem apresentar folhas maiores que o normal, além de bolsas de goma nos ramos novos e na casca dos frutos. Já a deficiência de manganês se caracteriza por folhas novas de tamanho quase normal, perda de brilho e clorose — uma tonalidade amarelada entre as nervuras, que permanecem verdes \cite{IntroduçãoGreening, IntroduçãoDeficiencias}.

Nosso projeto tem como objetivo analisar as deficiências de cobre e manganês nas folhas de Citrus reticulata. Essas duas condições foram selecionadas por apresentarem semelhanças com alguns dos sintomas do greening, como o amarelamento e o crescimento anormal das folhas. Utilizando a análise de imagens, buscamos identificar se uma folha apresenta sinais de uma dessas deficiências nutricionais. Com base no diagnóstico, o agricultor poderá adotar as medidas corretivas adequadas, como a verificação do estado nutricional do solo e a aplicação de fertilizantes de forma precisa e equilibrada, contribuindo para a saúde da planta e a maximização da produtividade.

Atualmente, a Inteligência Artificial (IA) é utilizada em diversas áreas do mercado de trabalho, desde o setor de saúde até os anúncios que vemos diariamente. Ela tem sido aplicada para aumentar a eficiência e a qualidade, proporcionando resultados mais precisos em uma ampla gama de situações. O objetivo da IA é operar de forma autônoma, oferecendo respostas exatas para cada contexto, sem a necessidade de intervenção humana. Assim, a adoção da IA em grandes empresas, hospitais e em outros setores que buscam automação está se tornando cada vez mais comum e significativa.

O sistema desenvolvido será um aplicativo móvel que utilizará Inteligência Artificial (IA) para identificar deficiências de manganês e cobre nas folhas de mexerica. Com essa ferramenta, o usuário poderá apontar a câmera do celular para a folha, permitindo que a IA analise possíveis sinais de deficiência, como clorose ou uma coloração verde intensa. Com base nessa análise, o sistema indicará se a folha apresenta sintomas de deficiência nutricional ou se está saudável, proporcionando um diagnóstico prático e acessível diretamente no campo.

A parte da Inteligência Artificial (IA) que nosso projeto utilizará será a visão computacional, que permite que a máquina "veja" visualmente e identifique diversos dados a partir das imagens. Para aplicar a visão computacional, são necessárias duas tecnologias-chave: redes neurais convolucionais e aprendizado de máquina profundo, conhecido como Deep Learning. Essas ferramentas possibilitam que a máquina reconheça padrões específicos. A rede neural convolucional é utilizada em contextos que exigem visão computacional e é inspirada na organização hierárquica do córtex visual humano, sendo composta por camadas interconectadas de neurônios \cite{IntroduçãoRedeNeural}. Por outro lado, o Deep Learning é uma evolução do aprendizado de máquina, consistindo em uma rede neural com múltiplas camadas. Com o Deep Learning, basta indicar se uma imagem contém uma folha de Citrus reticulata saudável ou com deficiência, e, ao fornecer muitas imagens rotuladas, a máquina aprenderá a identificar os padrões. Em contraste, no aprendizado de máquina tradicional, é necessário descrever de forma programática como é a folha saudável, especificando as características dos pixels na imagem do padrão que se deseja que a máquina reconheça \cite{IntroduçãoDeepLearning}.