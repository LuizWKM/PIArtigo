A agricultura de citros lida com diversos tipos de doenças nesta família, que são identificados a partir de observação de deformidades físicas e de pigmento em suas folhas, em sua casca e no fruto, além da presença de pragas no ambiente que afetam a qualidade final do alimento. A análise foliar é uma técnica comum usada para diagnosticar deficiências nutricionais em plantas, consistindo na coleta de folhas da planta, seguida pela análise química para determinar a concentração de determinados nutrientes. No entanto, por ser um processo que pode ser demorado e requer habilidades técnicas especializadas, e a utilização de IA na busca de padrões visuais facilitaria na determinação desse diagnóstico.

O projeto possui como objetivo o desenvolvimento de um sistema que seja capaz de reconhecer a deficiência dos nutrientes manganês e cobre da mexerica, a partir da análise de fotos de manchas em suas folhas com o auxílio da Inteligência Artificial, que através de padrões de identificação facilitará um possível diagnóstico diretamente para o agricultor produtor de mexerica. Os objetivos específicos são:
\begin{enumerate}
\item\textbf{Aprofundar na coleta de informações sobre os sintomas das deficiências de nutrientes específicos da mexerica.}
\item\textbf{Desenvolver um banco de imagens para estudo com uma variedade de níveis de deficiência de manganês e cobre para análise do problema;} 
\item\textbf{Realizar o treinamento da IA a partir do sistema de arquitetura de rede Neural Convolucional (CNNs).}
\item\textbf{Implementar um sistema que permita a identificação tecnicizada da deficiência usando Inteligência Artificial a partir de imagens digitais tiradas pelo próprio agricultor.}
\end{enumerate}