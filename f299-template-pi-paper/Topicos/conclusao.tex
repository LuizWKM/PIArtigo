Inicialmente, o projeto tem por objetivo buscar uma solução para as deficiências de minerais encontradas na mexeriqueira (Citrus reticulata), especificamente na variedade conhecida como mexerica, e, consequentemente, apresentar uma alternativa ao problema utilizando tecnologia. Dessa forma, elaboramos um projeto cuja ideia central é criar uma aplicação capaz de identificar, em tempo real, a deficiência de minerais pela qual a planta está passando. Isso permitirá produzir um diagnóstico preciso, facilitando o processo de recuperação da planta, evitando a perda de frutos e otimizando o tempo dos produtores.

Diante disso, nos deparamos com algumas limitações ao longo da pesquisa, pois ela é restrita a apenas alguns países e às condições do solo. A falta de minerais na planta possui relação direta com a acidez do solo. Nosso estudo tem por objetivo identificar especificamente deficiências de manganês e cobre em plantas de Citrus reticulata (mexerica) através da análise foliar. Pesquisas mostram que uma planta com deficiência de manganês geralmente não apresenta deficiência de cobre e vice-versa. Uma análise mais aprofundada revela que a deficiência está diretamente ligada ao pH do solo. A carência de manganês pode ser identificada em um solo mais ácido, enquanto a deficiência de cobre pode se manifestar em um solo alcalino \cite{ConclusãoMicroN, ConclusãoCobre}.