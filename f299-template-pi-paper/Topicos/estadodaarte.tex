Entre os muitos desafios enfrentados pelos agricultores, a deficiência de minerais nas plantas é uma preocupação significativa, pois pode resultar em perdas de produtividade e qualidade dos cultivos. A mexerica (Citrus reticulata) é uma das culturas suscetíveis a deficiências minerais, o que pode afetar seu crescimento, desenvolvimento e produção.

Recentemente, a inteligência artificial (IA) está sendo uma ferramenta poderosa para a detecção e diagnóstico de doenças e deficiência minerais em plantas. Com o uso de técnicas de aprendizado de máquina, visão computacional e análise de dados, surgiram aplicativos como o Plantix um aplicativo gratuito, desenvolvido por pesquisadores alemães, que faz a identificação de doenças em vários tipos de culturas com o uso da IA, o aplicativo foi referenciado no artigo \textcite{EstadoArte1}, como uma forma de identificar possíveis fitopatologias, que foi justamente a proposta do sistema que ele propôs. Um aplicativo mobile que auxilia na identificação de fitopatologias em diversas culturas, através de inteligência artificial.

Apesar de diversas semelhanças com o nosso projeto, há diferenças de objetivo, com o NitrusLeaf sendo focado na identificação de ausência de minerais na planta de mexerica através de uma analise por IA das folhas da planta, e a do artigo \textcite{EstadoArte1} sendo focado na identificação de fitopatologias em diversas culturas.

Pesquisadores têm utilizado câmeras de alta resolução e técnicas avançadas de processamento de imagens. Essas imagens são então analisadas por algoritmos de visão computacional para identificar padrões característicos associados a diferentes funções. O projeto de \textcite{EstadoArte2} busca fazer uma classificação de laranjas através de visão computacional para auxiliar na separação de laranjas ao comércio, invés do uso de peneiras que fazem a separação de laranjas grandes e pequenas. O sistema teve uma taxa de exatidão de 82\%, errando bastante na classificação de laranjas médias, porem mostra que a tecnologia tem uma grande eficácia. 

Já o nosso projeto é mais focado no cultivo e tratamento da planta, enquanto o projeto de \textcite{EstadoArte2} tem um foco no setor econômico e do comércio de laranjas, porem o nosso projeto também pode refletir nesses setores através de uma consequência dos nossos resultados, já que as árvores terão a produtividade esperada. Além do uso de visão computacional na identificação de deficiências minerais, que foi a mesma tecnologia utilizado no projeto de \textcite{EstadoArte2}, e seus resultados mostram que a eficácia dessa tecnologia é muito promissora.

O projeto de \textcite{EstadoArte3} utiliza também visão computacional com aprendizagem profunda, onde a proposta deles é detectar e classificar imagens de frutas, com o objetivo de fornecer informações nutricionais precisas com o uso de tabelas de composição alimentar. Utilizando um modelo inteligente baseado em CNN (Convolutional Neural Network) e Deep Learning. 

O projeto teve resultados excelentes com uma média de taxa de acerto de 96,5\% na identificação de bananas, maçãs e laranjas.

Com esses resultados mostra que um bom uso das tecnologias citadas, podemos chegar em um resultado satisfatório com o nosso projeto, onde grande parte dessas tecnologias vão ser a base do nosso projeto que busca, diferentemente dos outros projetos, dar uma função bastante complexa a máquina, que é a identificação da deficiência de minerais como o manganês e o cobre através da análise de imagens entregues pelos usuários.