A agricultura de citros enfrenta diversos tipos de doenças, que podem ser identificadas a partir da observação de deformidades físicas e alterações de pigmentação em suas folhas, cascas e frutos, além da presença de pragas que afetam a qualidade final do alimento. A análise foliar é uma técnica comum utilizada para diagnosticar deficiências nutricionais em plantas, que consiste na coleta de folhas seguida pela análise química para determinar a concentração de determinados nutrientes. No entanto, esse processo pode ser demorado e requer habilidades técnicas especializadas. A utilização de inteligência artificial (IA) para buscar padrões visuais pode facilitar a determinação desse diagnóstico.

O projeto tem como objetivo desenvolver um sistema capaz de reconhecer deficiências de manganês e cobre em folhas de mexerica a partir da análise de fotos de manchas, utilizando IA para facilitar um diagnóstico direto ao agricultor. Os objetivos principais são:
\begin{enumerate} 
\item \textbf{Aprofundar na coleta de informações sobre os sintomas das deficiências nutricionais específicas da mexerica.} 
\item \textbf{Desenvolver um banco de imagens com uma variedade de níveis de deficiência de manganês e cobre para análise do problema.} \item \textbf{Realizar o treinamento da IA utilizando uma arquitetura de rede Neural Convolucional (CNNs).} 
\item \textbf{Implementar um sistema que possibilite a identificação técnica das deficiências por meio de Inteligência Artificial, utilizando imagens digitais capturadas pelos próprios agricultores.} 
\end{enumerate}

Além dos objetivos principais, este projeto também busca atingir os seguintes objetivos específicos:

\begin{enumerate} 
\item \textbf{Desenvolver um sistema de visão computacional para identificar sintomas de deficiências nutricionais em imagens de folhas de mexericas, permitindo uma análise detalhada e precisa das condições de saúde da planta.}
\item \textbf{Criar uma funcionalidade que permita ao usuário analisar a folha da mexerica ao apontar a câmera do celular para a planta, possibilitando que a IA identifique e interprete o estado de saúde da planta em tempo real.}
\item \textbf{Após a análise do estado da planta, habilitar o cadastro do diagnóstico pelo usuário, com a opção de registrar o talhão e o número da planta para facilitar o acompanhamento e controle.} 
\item \textbf{Implementar uma funcionalidade de histórico para monitoramento, permitindo que o usuário visualize a evolução do estado das plantas e dos talhões ao longo do tempo, com a possibilidade de comparar estados anteriores e atuais.}
\item \textbf{Desenvolver, se possível, uma funcionalidade para a captura de imagens da plantação por drones, possibilitando a identificação de áreas mais ou menos saudáveis, facilitando a detecção de plantas potencialmente doentes.}
\item \textbf{Incluir um mapa interativo que mostre a localização de cada planta nos talhões, com legendas que indiquem a situação de saúde, número da planta e do talhão.}
\item \textbf{Criar um mapa de calor para destacar áreas críticas, com cores representando diferentes situações, como deficiência de manganês, deficiência de cobre e outras deficiências, facilitando a visualização dos pontos de foco e concentração de problemas.}
\item \textbf{Realizar uma análise de precisão do sistema desenvolvido, comparando-o com os métodos tradicionais de análise de solo e folhas, para validar a eficácia do sistema.} 
\item \textbf{Fornecer recomendações práticas e úteis com base nos resultados obtidos, para que os agricultores possam tratar de forma direcionada as deficiências detectadas, caso necessário.} \end{enumerate}