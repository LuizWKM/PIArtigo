\section{Metodologia}

O desenvolvimento do projeto será conduzido em etapas que seguem a metodologia de desenvolvimento ágil (Scrum), permitindo uma adaptação flexível aos requisitos ao longo do processo. As etapas principais incluem:

\subsection{Coleta de requisitos e design}
Usaremos o Figma para desenvolver protótipos de baixa e alta fidelidade das interfaces da aplicação, garantindo uma visualização clara dos requisitos funcionais e de usabilidade.

\subsection{Desenvolvimento front-end}
O código será estruturado com HTML e CSS para a definição de layout e estilo, enquanto o JavaScript será utilizado para proporcionar interatividade. O \textit{Node.js} será empregado para a criação de componentes reutilizáveis, melhorando a eficiência e a modularidade do código.

\subsection{Desenvolvimento back-end e banco de dados}
O banco de dados será modelado no \textit{brModelo} e implementado no \textit{MySQL} para garantir a correta estruturação dos dados. A interação entre \textit{front-end} e \textit{back-end} será implementada usando \textit{Python} para a integração da visão computacional e \textit{Node.js} no computador, enquanto o \textit{React Native} será utilizado para criar uma API no \textit{mobile}, com foco na criação de uma API eficiente.

\subsection{Testes e validação}
Após a implementação, serão realizados testes automatizados e manuais para validar o funcionamento correto de cada parte do sistema. Ferramentas como Selenium poderão ser utilizadas para automatizar os testes de interface.

\subsection{Publicação e acompanhamento}
O sistema será implementado na plataforma Web, sendo feito com \textit{front-end} e \textit{back-end} primeiramente, e em uma versão futura, algumas das funções disponíveis na Web serão adaptadas para \textit{mobile}, com o objetivo inicial de permitir o uso do aplicativo no campo para escanear a folha em tempo real e receber o feedback instantâneo. A fácil escalabilidade e manutenção serão garantidas, e ajustes serão feitos com base no feedback dos usuários após a implantação.