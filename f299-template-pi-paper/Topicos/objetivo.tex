A agricultura de citros enfrenta diversos tipos de doenças, que podem ser identificadas a partir da observação de deformidades físicas e alterações de pigmentação em suas folhas, cascas e frutos, além da presença de pragas que afetam a qualidade final do alimento. A análise foliar é uma técnica comum utilizada para diagnosticar deficiências nutricionais em plantas, que consiste na coleta de folhas seguida pela análise química para determinar a concentração de determinados nutrientes. No entanto, esse processo pode ser demorado e requer habilidades técnicas especializadas. A utilização de inteligência artificial (IA) para buscar padrões visuais pode facilitar a determinação desse diagnóstico.

O projeto tem como objetivo desenvolver um sistema capaz de reconhecer deficiências de manganês e cobre em folhas de mexerica a partir da análise de fotos de manchas, utilizando IA para facilitar um diagnóstico direto ao agricultor. Os objetivos principais são:
\begin{enumerate} 
\item \textbf{Aprofundar na coleta de informações sobre os sintomas das deficiências nutricionais específicas da mexerica.} 
\item \textbf{Desenvolver um banco de imagens com uma variedade de níveis de deficiência de manganês e cobre para análise do problema.} \item \textbf{Realizar o treinamento da IA utilizando uma arquitetura de rede Neural Convolucional (CNNs).} 
\item \textbf{Implementar um sistema que possibilite a identificação técnica das deficiências por meio de Inteligência Artificial, utilizando imagens digitais capturadas pelos próprios agricultores.} 
\end{enumerate}

Além dos objetivos principais, este projeto também busca atingir os seguintes objetivos específicos:

\begin{enumerate} 
\item \textbf{Desenvolver um sistema de visão computacional para identificar sintomas de deficiências nutricionais em imagens de folhas de mexericas.} 
\item \textbf{Realizar uma análise de precisão do sistema desenvolvido em comparação com análises tradicionais de solo e folhas.} 
\item \textbf{Fornecer recomendações práticas e úteis baseadas nos resultados obtidos, permitindo que os agricultores tratem as deficiências, quando necessário.} \end{enumerate}